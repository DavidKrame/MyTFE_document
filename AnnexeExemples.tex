\begin{center}
\setlength{\arrayrulewidth}{1pt}
\definecolor{gris}{gray}{0.65}
\begin{longtable}{|p{3cm}|p{13cm}|}
\caption{Tableau d'exemples de résumés} \label{myResumeLongTable} \\

\hline \cellcolor{gris} \textbf{Modèles} & \cellcolor{gris} \textbf{Résumés} \\ \hline \hline 
\endfirsthead

\multicolumn{2}{c}%
{{\bfseries \tablename\ \thetable{} \textbf{-- suite à partir de la page précédente}}} \\
\hline \cellcolor{gris} \textbf{Modèles} & \cellcolor{gris} \textbf{Résumés} \\ \hline \hline
\endhead

\hline \multicolumn{2}{|c|}{{\textbf{... $ \Downarrow $ ...}}} \\
\endfoot

\hline \hline
\endlastfoot
\multicolumn{2}{|c|}{{\textbf{Adresse à la nation (P.E. LUMUMBA, le 30 Juin 1960)\cite{broquet2008discours100}}}}\\
\hline
\textit{BART} & La République du Congo a été proclamée et notre cher pays est désormais entre les mains de ses propres enfants. Nous sommes fiers de cette lutte, qui a été de larmes, de feu et de sang, jusqu'au plus profond de nous-mêmes. Le nouveau Congo, notre chère République que créera mon gouvernement, sera un pays riche, libre et prospère. Nous allons faire régner non pas la paix des fusils et des baïonnettes, mais la paix des cœurs et la bonne volonté \\
\hline
\textit{BARThez} & Au nom du gouvernement congolais, je vous demande de faire de ce 30 juin 1960 une date illustre que vous garderez i\-ne\-ffa\-ça\-ble\-ment gravée dans vos cœurs, une date dont vous enseignerez avec fierté la signification à vos enfants. Nous avons connu le travail harassant, exigé en échange de salaires qui ne nous permettaient ni de manger à notre faim, ni de nous vêtir ou nous loger décemment, ni d'élever nos enfants comme des êtres chers. Nous avons connu les souffrances atroces des relégués pour opinions politiques ou croyances religieuses ; exilés dans leur propre patrie, leur sort était vraiment pire que la mort elle-même.
« Nous avons souffert dans notre corps et dans notre cœur de l'oppression colonialiste, nous vous le disons tout haut, tout cela est désormais fini. Le président de la République démocratique du Congo (RDC) Joe Biden a annoncé mercredi qu'il allait faire du Congo le centre de rayonnement de l'Afrique tout entière. Il a également annoncé que le pays allait revoir toutes les lois qui étaient en vigueur. Le président de la République française, François Hollande, prononce un discours liminaire à l'Assemblée nationale. « Mon pays, la France, sera prête à vous accorder son aide et son amitié, dit le président français. Dans un entretien fleuve, le président de la République démocratique du Congo a demandé à ses compatriotes de ne pas baisser les bras et de continuer à travailler pour que le Congo soit un pays riche, libre et prospère. Au nom du gouvernement, je vous demande de respecter la vie et les biens de vos concitoyens et des étrangers établis dans notre pays. Au nom de la République démocratique du Congo, l'Assemblée générale de l'Organisation des nations unies (ONU) a adopté un vigoureux texte appelant les Congolais à travailler pour créer une économie nationale. \\
\hline
\textit{Gensim} & Congolais et Congolaises,Combattants de l'indépendance aujourd'hui victorieux, Je vous salue au nom du gouvernement congolais, À vous tous, mes amis, qui avez lutté sans relâche à nos côtés, je vous demande de faire de ce 30 juin 1960 une date illustre que vous garderez i\-ne\-ffa\-ça\-ble\-ment gravée dans vos cœurs, une date dont vous enseignerez avec fierté la signification à vos enfants, pour que ceux-ci à leur tour fassent connaître à leur fils et à leurs petits-fils l'histoire glorieuse de notre lutte pour la liberté.Car cette indépendance du Congo, si elle est proclamée aujourd'hui dans l'entente avec la Belgique, pays ami avec qui nous traitons d'égal à égal, nul Congolais digne de ce nom ne pourra jamais oublier cependant que c'est par la lutte qu'elle a été conquise(applaudissements), une lutte de tous les jours, une lutte ardente et idéaliste, une lutte dans laquelle nous n'avons ménagé ni nos forces,ni nos privations, ni nos souffrances, ni notre sang.Cette lutte, qui fut de larmes, de feu et de sang, nous en sommes fiers jusqu'au plus profond de nous-mêmes, car ce fut une lutte noble et juste, une lutte indispensable pour mettre fin à l'humiliant esclavage qui nous était imposé par la force.Ce que fut notre sort en quatre-vingts ans de régime colonialiste,nos blessures sont trop fraîches et trop douloureuses encore pour que nous puissions les chasser de notre mémoire. \\
\hline
\textit{Merging} & Congolais et Congolaises, Combattants de l'indépendance aujourd'hui victorieux, Je vous salue au nom du gouvernement congolais, À vous tous, mes amis, qui avez lutté sans relâche à nos côtés, je vous demande de faire de ce 30 juin 1960 une date illustre que vous garderez i\-ne\-ffa\-ça\-ble\-ment gravée dans vos cœurs, une date dont vous enseignerez avec fierté la signification à vos enfants, pour que ceux-ci à leur tour fassent connaître à leur fils et à leurs petits-fils l'histoire glorieuse de notre lutte pour la liberté. Car cette indépendance du Congo, si elle est proclamée aujourd'hui dans l'entente avec la Belgique, pays ami avec qui nous traitons d'égal à égal, nul Congolais digne de ce nom ne pourra jamais oublier cependant que c'est par la lutte qu'elle a été conquise(applaudissements), une lutte de tous les jours, une lutte ardente et idéaliste, une lutte dans laquelle nous n'avons ménagé ni nos forces, ni nos privations, ni nos souffrances, ni notre sang. Mais pour que nous arrivions sans retard à ce but, vous tous, législateurs et citoyens congolais, je vous demande de m'aider de toutes vos forces. J'invite tous les citoyens congolais, hommes, femmes et enfants, de se mettre résolument au travail en vue de créer une économie nationale prospère qui consacrera notre indépendance économique. \\
\hline
\textit{BARTkrame-abs} & Nous avons connu le travail harassant, exigé en échange de salaires Nous avons connu les souffrances atroces des relégués pour opinions Ensemble, mes frères, mes sœurs, nous allons commencer une nouvelle lutte, Nous allons montrer au monde ce que peut faire l'homme noir quand il travaille dans Nous allons régner non pas la paix des fusils et des baïonnettes Je vous demande à tous de ne reculer devant aucun sacrifice pour assurer la réussite Je vous demande de respecter inconditionnellement la vie et les biens de vos J’invite tous les citoyens congolais, hommes, femmes et enfants \\
\hline
\hline
\multicolumn{2}{|c|}{{\textbf{I have a dream (M.LUTHER KING, le 28 août 1963)\cite{broquet2008discours100}}}}\\
\hline
\textit{BART} & Cent ans se sont écoulés et l'existence du nègre est toujours entravée par les liens de la ségrégation. L'Amérique a émis aux Noirs un chèque sans valeur; un chèque qui est revenu avec la mention « fonds insuffisants » Ils ont compris que leur liberté est inextricablement liée à notre liberté. Et si l'Amérique veut être une grande nation, cela doit se réaliser. \\
\hline
\textit{BARThez} & Il y a un siècle, un grand Américain signait notre acte d'émancipation. Aujourd'hui, cent ans ont passé et le Noir n'est pas encore libre. L'Amérique a failli à sa promesse en ce qui concerne ses citoyens de couleur. C'est pourquoi nous sommes montés à la capitale de notre pays pour toucher un chèque. Nous sommes venus en ce lieu sanctifié pour rappeler à l'Amérique les exigeantes urgences de l'heure présente.Nous sommes venus encaisser un chèque qui nous fournira les richesses de la liberté et la sécurité de la justice. Dans un discours prononcé en 1963 à l'Assemblée nationale, George Floyd déclare qu'il n'y aura « plus ni repos ni tranquillité en Amérique tant que le Noir n'aura pas obtenu ses droits de citoyen ».
Comme le relaye le magazine américain The Conversation, « Black Lives Matter », le mouvement de libération de la communauté noire appelle à la vigilance face à la violence physique. « Nous ne pourrons jamais être satisfaits tant que nos corps recrus de la fatigue du voyage ne trouveront pas un abri dans les motels des grand-routes ou les hôtels des villes. » C'est la devise de la lutte pour les droits civiques aux Etats-Unis. « Nous ne pourrons être satisfaits tant qu'un Noir du Mississippi ne pourra pas voter. Nous ne serons pas satisfaits tant que le droit ne jaillira pas comme les eaux et la justice comme un torrent intarissable.« Ne nous vautrons pas dans les vallées du désespoir ». Le jour où les enfants de Dieu pourront chanter ensemble l'hymne américain, ils donneront une signification nouvelle : « Mon pays c'est toi. « Libres enfin. Merci Dieu tout-puissant, nous voilà libres enfin. » C'est la devise du mouvement de libération des Noirs américains. \\
\hline
\textit{Gensim} & Je suis heureux de participer avec vous aujourd'hui à ce rassemblement qui restera dans l'Histoire comme la plus grande manifestation que notre pays ait connue en faveur de la liberté.Il y a un siècle de cela, un grand Américain qui nous couvre aujourd'hui de son ombre symbolique signait notre acte d'émancipation. En ce sens, nous sommes montés à la capitale de notre pays pour toucher un chèque.En traçant les mots magnifiques qui forment notre constitution et notre Déclaration d'indépendance, les architectes de notre république signaient une promesse dont héritait chaque Américain.Aux termes de cet engagement, tous les hommes, les Noirs, oui,aussi bien que les Blancs, se verraient garantir leurs droits inaliénables à la vie, à la liberté et à la recherche du bonheur.Il est aujourd'hui évident que l'Amérique a failli à sa promesse en ce qui concerne ses citoyens de couleur. \\
\hline
\textit{Merging} & Je suis heureux de participer avec vous aujourd'hui à ce rassemblement qui restera dans l'Histoire comme la plus grande manifestation que notre pays ait connue en faveur de la liberté.Il y a un siècle de cela, un grand Américain qui nous couvre aujourd'hui de son ombre symbolique signait notre acte d'émancipation. » Et si l'Amérique doit être une grande nation, cela doit devenir vrai.Aussi, que résonne la liberté depuis les prodigieux sommets du NewHampshire. Qu'elle résonne depuis les puissantes montagnes de l'État de New York. Qu’elle résonne depuis les hautes Alleghany de Pennsylvanie. Qu'elle résonne depuis les sommets neigeux des Rocheuses du Colorado. Qu'elle résonne depuis les pentes capricieuses de la Californie.Mais cela ne suffit pas. Qu'elle résonne depuis la Stone Mountain de Georgie. Qu'elle résonne depuis la Lookout Mountain du Tennessee. Qu'elle résonne depuis chaque colline et chaque butte du Mississippi, qu'elle résonne du flanc de chaque montagne.Quand nous ferons en sorte que la liberté puisse sonner, quand nous la laisserons carillonner depuis chaque village et chaque hameau, depuis chaque État et chaque cité, nous pourrons hâter la venue du jour où tous les enfants de Dieu, les Noirs et les Blancs,les Juifs et les Gentils, les catholiques et les protestants, pourront se tenir par la main et chanter les paroles du vieux negro spiritual : «Libres enfin.  \\
\hline
\textit{BARTkrame-abs} & Cent ans ont passé et le Noir n’est pas encore libre. Cent ans
Nous sommes montés à la capitale de notre pays pour toucher un chèque. En
Nous refusons de croire que la banque de la justice ait fait faillite. Nous
Il n’y aura plus ni repos ni tranquillité en Amérique tant que
Livrons toujours notre bataille sur les hauts plateaux de la dignité et
Nous ne pourrons jamais être satisfaits tant que nos enfants seront dépouillés
Nous ne pourrons être satisfaits tant qu’un Noir du Mississippi ne pourra
Je suis désolé, mais même si nous devons affronter des difficultés aujourd
» Je rêve que mes quatre petits enfants vivront un jour dans un pays où on
Avec une telle foi nous serons capables de transformer la cacophonie
Ce sera le jour où les enfants de Dieu pourront chanter ensemble cet hymne
Mais cela ne suffit pas. Qu'elle résonne depuis la Stone Mountain de \\
\hline
\hline
\multicolumn{2}{|c|}{{\textbf{Discours \textit{Ich bin ein Berliner} (J.F. KENNEDY, le 26 juin 1963)\cite{broquet2008discours100}}}}\\
\hline
\textit{BARThez} & ICH BIN EIN BERLINER. J’ai été invité par votre distinguéBourgmestre W. Brandt à venir visiter la République fédérale. « Lass’sie nach Berlinkommen ». C’est ce qu’a dit Donald Trump sur Twitter. Lors d'une visite à Berlin, le Bourgmestre de la ville a dit ne ressentir aucune satisfaction en voyant le mur qui divise Berlin entre ses habitants. Le pape François appelle l'humanité à se tourner vers la seule ville de Berlin ou de votre pays, l’Allemagne, vers le progrès de la liberté partout dans le monde. Tous les habitants de Berlin-Ouest pourront tirer une sobre satisfaction du faitqu’ils ont été sur la ligne de front pendant deux décennies.\\
\hline
\textit{BART} & Berlin-Ouest a conservé autant de vitalité, de force, d'espoir et de détermination que Berlin-Ouest. Le mur fournit la démonstration la plus éclatante de la faillite du système communiste. Une paix réelle et durable en Europe ne pourra jamais être assurée tant qu'un Allemand sur quatre sera privé du droit fondamental des hommes libres à l'autodétermination. \\
\hline
\textit{Gensim} & Je ne connais aucune ville assiégée durant dix-huit ans qqui ait conservé autant de vitalité, de force, d'espoir et de détermination que Berlin-Ouest. Le mur fournit la démonstration la plus éclatante de la faillite du système communiste et cette faillite est visible aux yeux du monde entier. Mais nous n'éprouvons aucune satisfaction en voyant ce mur,car il constitue, comme l'a dit votre Bourgmestre, une offense non seulement à l'histoire mais aussi à l'humanité, séparant les familles,les maris de leurs femmes et les frères de leurs soeurs, et divisant un peuple qui voudrait vivre uni.Ce qui est vrai pour cette ville l'est pour toute l'Allemagne. \\
\hline
\textit{Merging} & Je suis fier d'être venu dans votre ville, invité par votre distingué Bourgmestre [W. Brandt], qui a symbolisé aux yeux du monde entier l'esprit combatif de Berlin-Ouest. En conclusion, je vous demande donc de regarder au-delà des dangers d'aujourd'hui vers les espoirs de demain, au-delà de la seule liberté de cette ville de Berlin ou de votre pays, l'Allemagne, vers le progrès de la liberté partout dans le monde, au-delà de vous-mêmes et de nous-mêmes vers l'humanité tout entière. La liberté est indivisible et tant qu'un seul homme se trouve en esclavage, tous les autres ne peuvent se considérer comme libres.Mais quand tous les hommes seront libres, nous pourrons attendre avec impatience le jour où cette ville sera réunifiée, de même que ce pays et ce grand continent qu'est l'Europe dans un monde de paix et d'espoir.  \\
\hline
\hline
\multicolumn{2}{|c|}{{\textbf{Biographie d'Albert Einstein (Copie brute de la page \cite{WikiEins})}}}\\
\hline
\textit{BART} & L'intérêt d'Albert Einstein pour la science a été éveillé par une boussole lorsqu'il avait cinq ans. En 1901, il publie son premier article scientifique dans les Annalen der Physik, article consacré à ses recherches sur la capillarité. En 1922, il reçoit le prix Nobel de physique pour ses contributions à la physique théorique. Albert Einstein se lie d'amitié avec le chanteur noir Paul Robeson et devient, à ses côtés, un militant des droits civiques et de la lutte contre le racisme. Ben Gourion lui propose la présidence de l'Etat d'Israël en 1952, ce qu'il refuse. Pendant la guerre froide, il s'est prononcé contre la course aux armements et a appelé les scientifiques à renoncer ensemble à la prolifération des armes atomiques. La citation apocryphe qui lui est attribuée est : « L'astrologie est une science en soi, éclairante. Les connaissances géophysiques mettent en évidence le pouvoir des étoiles et des planètes sur le destin de la Terre. A son tour, dans un certain sens, l'astrologie la renforce. C'est pourquoi c'est une sorte d'élixir de vie pour l'humanité. L'expression a été inventée environ cinq ans après la mort d'Einstein.\\
\hline
\hline
\multicolumn{2}{|c|}{{\textbf{Biographie de $ 11 $ pages sur Gandhi accessible au lien \cite{TPsurGandhi}}}}\\
\hline
\textit{BART} & L'idéologie non violente de Gandhi revient à rejeter tout ce qui détruit la personne humaine. Martin Luther King, Nelson Mandela et Ernst Friedrich Schumacher sont tous de grandes figures de la non-violence. La pensée de Gandhi n'est pas explicitement ou directement appliquée par ceux qui prétendent le suivre, mais elle est toujours présente aujourd'hui.\\
\hline
\hline

\end{longtable}
\end{center}
De ces quelques exemples on peut remarquer que \textit{BAThez} est trop biaisé et insère souvent des bouts de textes et des auteurs n'ayant aucun rapport avec son contexte. Mais ces bouts insérés ont vrai\-sem\-bla\-ble\-ment un grand rapport avec le \textit{dataset} \textit{OrangeSum} utilisé pour entraîner le modèle \textit{BARThez}.
\subsection*{\underline{Résumé de ce mémoire par BART optimisé (ratio=0.3) sans retouche :}}
A l'ère du numérique, le texte est l'un des principaux moyens de communication et surtout de transmission des savoirs. En 2018, c'était environ $ 80\% $ des informations circulant sur le web. L'évolution de l'informatique ne cesse de démontrer la possibilité de simplifier toujours grandement la vie humaine en automatisant de plus en plus de tâches. Les réseaux de neurones artificiels (ANN) sont un ensemble de neurones (artificiels) assemblés pour résoudre des tâches considérées comme nécessitant une certaine intelligence. Le neurone artificiel est un algorithme développé en s'inspirant du modèle théorique simplifié d'un neurone naturel. Les cellules LSTM sont utilisées à la place des cellules RNN con\-ven\-tion\-nelles (appelées vanille) pour permettre au réseau de traiter des séquences de plus en plus longues sans perte rapide d'informations. TextRank est un algorithme de résumé extractif, basé sur la théorie des graphes et inspiré de l'algorithme PageRank de Google. La formation des transformateurs est semi-supervisée. Il existe des résumés mono\-document et multi\-do\-cu\-ment selon le nombre de documents sources. Un résumé peut également être classé selon le public auquel il est destiné. Le système doit permettre de : - Synthétiser les textes qui lui sont fournis en entrée (saisis directement ou importés dans des fichiers .pdf, .docx et .txt non scannés ; - Obtenir des synthèses produites par plusieurs algorithmes et les évaluer ; - Stocker les couples document-synthèse ; - Faciliter la navigation des documents en mettant en évidence les parties saillantes ; - Permettre l'affinement d'un modèle de synthèse automatique. Nous voulons développer une plateforme capable de restituer les résumés des textes qui lui seront présentés. Nous devons également être en mesure d'identifier les personnes qui utiliseront l'API dans leurs systèmes (les développeurs dont nous avons parlé dans la section précédente), d'où la nécessité de les enregistrer, de leur donner la possibilité de s'authentifier et de générer une clé d'authentification. . Nous avons prévu 4 familles d'end-points réservés à tout ce qui concerne les vues de l'API (ses interfaces pour permettre l'authentification, la génération de clés et la documentation) Pour la synthèse abstraite, le choix vers des modèles de type BART a été inspiré par la nécessité de réaliser des synthèses abstraites mais pas trop pour ne pas s'éloigner du contenu du texte source. BART est moins lourd et plus maniable que le modèle T5, c'était le seul à considérer (car PEGASUS et T5 étaient ipso facto exclus). Pour la synthèse abstraite, celle qui se rapproche le plus de la synthèse humaine, nous avons utilisé des modèles de type BART, pour éviter de pousser l'abstraction trop loin comme le ferait PEGASUS, et en même temps éviter d'utiliser inutilement un modèle. Un ensemble de données de synthèse automatique que nous avons nommé for-ULPGL-Dissertation hébergera au fil du temps les données de synthèse collectées via la base de données de notre système.
\subsection*{\underline{Pour d'autres exemples}}
Pour voir d'autres exemples réalisés grâce à ce système, visiter le lien \cite{AllResumes}.