$ _{} $ $ _{} $ $ _{} $ $ _{} $ $ _{} $A l'ère du numérique, le texte est l'un des principaux moyens de communication et surtout de transmission des savoirs. En 2018, c'était environ $ 80\% $ des informations circulant sur le web. Mais, le temps étant une denrée rare, on voudrait avoir la possibilité d'accéder directement aux informations saillantes des textes, ou juste en avoir un aperçu global avant d'y consacrer du temps. D'où la nécessité d'un système de résumé automatique performant.
Dans ce travail, notre objectif est de mettre en place un système de résumé automatique performant. Pour cela, nous avons considéré à la fois le résumé abstractif et extractif, laissant le choix à l'utilisateur, selon ses besoins en information.
Pour le résumé extractif, nous avons considéré deux approches qui ont consisté à utiliser un module python dénommé \textit{gensim} pour la synthèse, puis un mélange de plusieurs algorithmes de synthèse que nous avons nommé \textit{merging}. Pour le résumé abstractif, nous avons opté pour les modèles du type \textit{transformer} car ce sont eux qui ont atteint jusqu'à présent les performances les plus élevées pour diverses tâches de traitement du langage naturel. Néanmoins, nous avons servi les \textit{transformers} utilisés pour la synthèse abstractive à travers un pipeline devant permettre l'amélioration des performances et l'augmentation du nombre de pages recevables.
A l'issu de nos expérimentations, nous avons trouvé que l'algorithme \textit{merging} était qualitativement plus performant que \textit{gensim}. Nous avons également enregistré une nette amélioration des résultats fournis par les \textit{transformers} une fois servis à travers le pipeline que nous avons proposé.
Ainsi, nous avons obtenu un système capable de restituer les résumés abstractifs ou extractifs des textes. Pour cela, nous avons mis en place une interface de pro\-gram\-ma\-tion des ap\-pli\-ca\-tions que les développeurs pourraient utiliser pour réaliser la syn\-thè\-se au\-to\-ma\-ti\-que dans leurs systèmes respectifs, nous avons im\-plé\-men\-té une application web qui exploite les services de cette interface mise au point et avons réalisé un modèle qui sera entrain d'être amélioré au fil de la collecte des couples texte-synthèse à travers notre système.\\
\underline{\textbf{Mots-clés}} : Intelligence artificielle, Encodeur-décodeur, Transformer, résumé automatique des textes, Python, Flask, JavaScript, ReactJS, ExpressJS.